% \iffalse meta-comment
%<*internal>
\iffalse
%</internal>
%<*copyright>
Copyright 2023, RadioNoiseE
NianCLS, 年文檔類
%</copyright>
%<*internal>
\fi
\begingroup
\def\FORMAT{LaTeX2e}
\expandafter\endgroup
\ifx\fmtname\FORMAT\else
\csname fi\endcsname
%</internal>
%<*batchfile>
\input docstrip.tex
\keepsilent
\askonceonly
\edef\HEAD{\perCent\perCent 年文檔類 黄京}
\generate{
    \usepreamble\HEAD
    \usepostamble\empty
    \file{niancls.cls}{\from{\jobname.dtx}{cls}}
}
\begingroup
\obeyspaces
\Msg{*******}
\Msg{* End *}
\Msg{*******}
\endgroup
\endbatchfile
%</batchfile>
%<*internal>
\fi
%</internal>
%<*driver>
\makeatletter
\def\ltj@stdmcfont{SourceHanSerifSC}
%%\def\ltj@stdyokojfm{eva/{sc,nstd}}
\makeatother
\documentclass{ltjsarticle}
%%\documentclass{ctxdoc}
\usepackage{doc}
\EnableCrossrefs
\CodelineIndex
\RecordChanges
%%\def\PrintDescribeMacro{}
%%\def\PrintDescribeEnv{}
%%\def\PrintMacroName{}
%%\def\PrintEnvName{}
\usepackage{luatexja-fontspec}
%%\usepackage{fontspec}
\setmainfont{Linux Libertine O}
\setmainjfont{SourceHan Serif SC}[YokoFeatures={JFM={eva/{sc,nstd}}}]
\setsansfont{Linux Biolinum O}
%%\setmonofont{Iosevka Slab}[Scale=MatchLowercase, FakeStretch=1.138]
\setmonofont{Fira Code}[Scale=MatchLowercase]
\def\LuaTeX{Lua\kern-.2ex\TeX}
\begin{document}
\DocInput{\jobname.dtx}
\PrintIndex
\end{document}
%</driver>
% \fi
% \title{\sffamily\gtfamily 年文檔類{\quad}Nian{\it CLS}}
% \author{黄京}
% \date{西曆2023年}
% \maketitle
% \begin{abstract}
% 为在{\LuaTeX}下排印中日文本而作的文档类。基于{\sffamily expl3}构建。
% \end{abstract}
% \section{初始化}
% \subsection{載入{\LaTeX3}並檢驗依賴}
%    \begin{macrocode}
%<@@=ncls>
\NeedsTeXFormat{LaTeX2e}
\RequirePackage{expl3}
\ProvidesExplClass{niancls}{2023-03-24}{1.0.0}{Nian Document Class}
\prop_gput:Nnn \g_msg_module_name_prop { ncls } { niancls }
%    \end{macrocode}
% 申明结束。接下来检查依赖,首先为{\sffamily xparse}及{\sffamily l3keys2e}宏包。
%    \begin{macrocode}
\cs_if_exist:NF \NewDocumentCommand
  { \RequirePackage { xparse } }
\RequirePackage { l3keys2e }
%    \end{macrocode}
% 接下来检查{\sffamily expl3}的版本。
%    \begin{macrocode}
\@ifpackagelater { expl3 } { 2021-02-10 } { }
  {
    \msg_new:nnnn { ncls } { latex3-too-old }
      { Package~`l3kernel'~and~`l3packages'~too~old. }
      {
        You~need~to~update~your~installation~of~the~bundles~
        `l3kernel'~and~`l3packages'. \\
        Loading~niancls~will~abort!
      }
    \msg_critical:nn { ncls } { latex3-too-old }
  }
%    \end{macrocode}
% 后进行{\LaTeXe}格式之版本检查。
%    \begin{macrocode}
\@ifl@t@r \fmtversion { 2021-06-01 } { }
  {
    \msg_new:nnnn { ncls } { latex-too-old }
      { Format~LaTeX2e~version~too~old. }
      {
        You~need~to~update~your~LaTeX2e~to~the~latest~release. \\
        Loading~niancls~will~abort!
      }
    \msg_critical:nn { ncls } { latex-too-old }
  }
%    \end{macrocode}
% 最后检查{\LaTeX}引擎,仅支持使用{\LuaTeX}编译。
%    \begin{macrocode}
\sys_if_engine_luatex:TF { }
  {
    \msg_new:nnnn { ncls } { unsupported-engine }
      { LuaTeX~is~the~only~supported~engine~for~niancls. }
      {
        You~should~switch~to~LuaTeX~to~use~niancls. \\
        Loading~niancls~will~abort!
      }
    \msg_fatel:nn { ncls } { unsupported-engine }
  }
%    \end{macrocode}
% \subsection{私有定義}
% 定义\verb|\AtEndOfClass|钩子。
%    \begin{macrocode}
\cs_new_protected:Npn \@@_at_end:n { \AtEndOfClass }
%    \end{macrocode}
% 定义用于在读取结束后释放缓存的宏。
%    \begin{macrocode}
\seq_new:N \g_@@_aftercls_del_seq
\cs_set:Nn \@@_aftercls_addtodel:N
  { \seq_gput_right:Nn \g_@@_aftercls_del_seq { #1 } }
\@@_aftercls_addtodel:N \@@_aftercls_addtodel:N
\@@_at_end:n
  {
    \cs_undefine:N \g_@@_aftercls_del_seq
  }
%    \end{macrocode}
% \section{主要特性}
% \subsection{紙張配置}
% 申明存储纸张尺寸信息的特性列表。
%    \begin{macrocode}
\prop_new:N \g_@@_papersizelist_prop
\tl_new:N \g_@@_papersizeinfo_tl
%    \end{macrocode}
% 随后定义用于处理尺寸信息的宏。
%    \begin{macrocode}
\cs_set:Nn \@@_addpapersize:nnn
  {
    \prop_put_if_new:Nnn \g_@@_papersizelist_prop
      { #1 }
      { { #2 }, { #3 } }
  }
%    \end{macrocode}
% 于键对值列表中定义\verb|paper|键处理纸张信息之配置。
%    \begin{macrocode}
\keys_define:nn { ncls }
  {
    paper .code:n = \tl_gset:Nn \g_@@_papersizeinfo_tl { #1 },
    paper .default:n = a4paper,
    paper .value_required:n = false
  }
%    \end{macrocode}
% 处理并设置尺寸信息。
%    \begin{macrocode}
\prop_if_in:NnTF \g_@@_papersizelist_prop { 
\pdf_pagesize_gset:nn { } { }
%    \end{macrocode}
\endinput
